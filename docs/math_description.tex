\documentclass[11pt,a4paper]{article}

\usepackage[polish]{babel}
\usepackage[utf8]{inputenc}
\usepackage[T1]{fontenc}
\usepackage{lmodern}
\usepackage{indentfirst}
\usepackage{fullpage}
\usepackage{enumerate}
\usepackage{secdot}
\usepackage{verbatim}
\usepackage{graphicx}

\selectlanguage{polish}
\frenchspacing

% Zdefiniowanie autorów i~tytułu:
\author{
	Karol Dzitkowski\\
	Robert Jakubowski\\
	Tomasz Janiszewski
}
\title{
	Teoria Ramseya\\
	\huge{Ramsey Game - gra dwuosobowa}\\
	Teoretyczny opis problemu
 }

\begin{document}
\maketitle
\newpage

% Wstawienie spisu tresci:
\tableofcontents
\newpage

%Treść:
\section{Wstęp}
Grafy w swej charakterystyce dają duże możliwości. Ich natura sprawia, że mimo współczesnego zaawansowania matematyki, ciągle można w nich wiele odkryć. Jednym z popularno-naukowych odkryć jest gra Ramseya, która ma proste zasady, a jednocześnie jest ściśle związana z teorią grafów.

\section{Gra Ramseya}
W grze Ramseya biora udział 2 osoby.
Planszą jest graf, który składa się ze skończonej, ustalonej liczby wierzchołków tworząc graf pełny.
Każdy z graczy stara się stworzyć klikę o zadanej wielkości zaznaczając
na przemian krawędzie własnym kolorem. Wygrywa gracz, który pierwszy utworzy klikę o zadanej wielkości w swoim kolorze.

\section{Złożonośc problemu}

\subsection{Wielkość tworzonej kliki}
Dla kliki o wielkości 2 problem jest trywialny - wystarczy narysować jedną krawędź.
Klika 3-wierzchołkowa zawiera w sobie 3 krawędzie. Gra w takim układzie ciągle nie jest trudna do przeanalizowania.
Sytuacja się zmienia, gdy weżmiemy pod uwagę kliki 4-wierzchołkowe i 5-wierzchołkowe.
Już samo znalezienie takiej kliki w grafie nie jest zadaniem trywialnym, a co dopiero konkurowanie w jej tworzeniu.
Wzrost wielkości liczby krawędzi w klice od liczby jej wierzchołków jest kwadratowy i wyraża się wzorem: k = (n * (n - 1)) / 2

\subsection{Wielkość grafu wejściowego}
Skomplikowanie gry zależy też od wielkości grafu wejściowego.
Zwiększanie ilości wierzchołków będzie powodowało duży wzrost wariantów przebiegu gry.
Z góry tą wielkość można oszacować poprzez k! = [(n * (n - 1)) / 2]!
, gdzie k - liczba krawędzi, n - liczba wierzchołków. Przebieg gry można zapisywać jako ciąg wybieranych krawędzi.
W pierwszym ruchu gracz1 ma do wyboru k krawędzi. Gracz2 k-1 krawędzi. W drugim ruchu gracz1 k-2 krawędzi, a gracz2 k-3 krawędzi, itd.
Oczywiście tę wartość można zmniejszyć, ponieważ wiele gier ma przebieg symetryczny.
W szczególności nie jest istotne, która krawędź zostnie wybrana na początku gry.

\section{Minimalny rozmiar planszy}
O remisie w grze mówimy, gdy żadnemu z graczy nie udało się stworzyć kliki o zadanej wielkości.\\
Gra będzie remisowa zawsze, gdy rozmiar kliki jest większy niż rozmiar planszy.
Dówód jest dość oczywisty - gracz nie może stworzyć tej kilki, ponieważ nie dysponuje wystarczającą ilością wierzchołków.\\
Tę wartość jeszcze można zmniejszyć.
Należy pamiętać, że gracze wykonują ruchu na zmianę. Oznacza to, że gdy gracz1 wykonał n ruchów, to gracz2 wykonał przynajmniej n-1 ruchów.
Stworzenie kliku kładającej się z x wierchołków wymaga (x * (x-1)) / 2 krawędzi.
Tym samym, aby zakońćzyć grę na planszy musi pojawić się przynajmniej (x * (x-1)) - 1 krawędzi.
Dla planszy o y wierzchołkach mamy (y * (y - 1)) / 2 wierzchołków.
Oznacza to, że remis padnie zawsze, gdy (y * (y-1)) / 2 < (x * (x-1)) - 1\\
Ciekawe jest to, że taka sytuacja, gdzie nie może zostać wyłoniony zwycięzca zdarza się, gdy mamy planszę wielkości 5 oraz próbujemy stworzyć klikę wielkości 4.
\end{document}
