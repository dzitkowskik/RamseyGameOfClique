\documentclass[11pt,a4paper]{article}

\usepackage[polish]{babel}
\usepackage[utf8]{inputenc}
\usepackage[T1]{fontenc}
\usepackage{lmodern}
\usepackage{indentfirst}
\usepackage{fullpage}
\usepackage{enumerate}
\usepackage{secdot}
\usepackage{verbatim}
\usepackage{graphicx}

\selectlanguage{polish}
\frenchspacing

% Zdefiniowanie autorów i~tytułu:
\author{
	Karol Dzitkowski\\
	Robert Jakubowski\\
	Tomasz Janiszewski
}
\title{
	Teoria Ramseya\\
	\huge{Ramsey Game - gra dwuosobowa}\\
	Opis projektu
 }

\begin{document}
\maketitle
\newpage

% Wstawienie spisu tresci:
\tableofcontents
\newpage

%Treść:
\section{Temat projektu}
Projekt będzie realizował grę Ramseya, w której biorą udział 2 osoby.
Planszą jest graf, który składa się ze skończonej, ustalonej liczby wierzchołków połączonych ze sobą szarymi
krawędziami, tworząc graf pełny. Każdy z graczy stara się stworzyć klikę o zadanej wielkości zaznaczając
na przemian krawędzie własnym kolorem. Wygrywa gracz, który pierwszy utworzy klikę o zadanej wielkości w swoim kolorze.

\section{Aplikacja}
Aplikacja będzie składała się z dwóch widoków: widoku menu oraz widoku gry.

\subsection{Widok menu}
W tym widoku będą znajdowały się przyciski:

\begin{itemize}
	\item start - rozpoczyna grę,
	\item ustawiania wielkości kliki (wraz z polem do wpisania wielkości),
	\item ustawiania wielkości planszy (wraz z polem do wpisania wielkości),
	\item exit - wyjście z aplikacji.
\end{itemize}

\subsection{Widok gry}
Po rozpoczęciu rozgrywki pojawi się graf o ilości wierzchołków równej wielkości planszy, połączonych szarymi krawędziami.
Gracze na przemian będą wybierali krawędzie, które po kliknięciu przyciskiem myszy będą kolorowały
się na kolor gracza. W momencie utworzenia jednokolorowej kliki przez któregoś z graczy, wyświetlany
będzie alert z wiadomością, który z graczy wygrał i po zatwierdzeniu nastąpi przekierowanie do widoku menu.

\end{document}
